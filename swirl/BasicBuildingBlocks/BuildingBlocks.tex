% Options for packages loaded elsewhere
\PassOptionsToPackage{unicode}{hyperref}
\PassOptionsToPackage{hyphens}{url}
%
\documentclass[
]{article}
\usepackage{amsmath,amssymb}
\usepackage{lmodern}
\usepackage{iftex}
\ifPDFTeX
  \usepackage[T1]{fontenc}
  \usepackage[utf8]{inputenc}
  \usepackage{textcomp} % provide euro and other symbols
\else % if luatex or xetex
  \usepackage{unicode-math}
  \defaultfontfeatures{Scale=MatchLowercase}
  \defaultfontfeatures[\rmfamily]{Ligatures=TeX,Scale=1}
\fi
% Use upquote if available, for straight quotes in verbatim environments
\IfFileExists{upquote.sty}{\usepackage{upquote}}{}
\IfFileExists{microtype.sty}{% use microtype if available
  \usepackage[]{microtype}
  \UseMicrotypeSet[protrusion]{basicmath} % disable protrusion for tt fonts
}{}
\makeatletter
\@ifundefined{KOMAClassName}{% if non-KOMA class
  \IfFileExists{parskip.sty}{%
    \usepackage{parskip}
  }{% else
    \setlength{\parindent}{0pt}
    \setlength{\parskip}{6pt plus 2pt minus 1pt}}
}{% if KOMA class
  \KOMAoptions{parskip=half}}
\makeatother
\usepackage{xcolor}
\IfFileExists{xurl.sty}{\usepackage{xurl}}{} % add URL line breaks if available
\IfFileExists{bookmark.sty}{\usepackage{bookmark}}{\usepackage{hyperref}}
\hypersetup{
  pdftitle={R Basic Building Blocks},
  pdfauthor={swirl Team},
  hidelinks,
  pdfcreator={LaTeX via pandoc}}
\urlstyle{same} % disable monospaced font for URLs
\usepackage[margin=1in]{geometry}
\usepackage{color}
\usepackage{fancyvrb}
\newcommand{\VerbBar}{|}
\newcommand{\VERB}{\Verb[commandchars=\\\{\}]}
\DefineVerbatimEnvironment{Highlighting}{Verbatim}{commandchars=\\\{\}}
% Add ',fontsize=\small' for more characters per line
\usepackage{framed}
\definecolor{shadecolor}{RGB}{248,248,248}
\newenvironment{Shaded}{\begin{snugshade}}{\end{snugshade}}
\newcommand{\AlertTok}[1]{\textcolor[rgb]{0.94,0.16,0.16}{#1}}
\newcommand{\AnnotationTok}[1]{\textcolor[rgb]{0.56,0.35,0.01}{\textbf{\textit{#1}}}}
\newcommand{\AttributeTok}[1]{\textcolor[rgb]{0.77,0.63,0.00}{#1}}
\newcommand{\BaseNTok}[1]{\textcolor[rgb]{0.00,0.00,0.81}{#1}}
\newcommand{\BuiltInTok}[1]{#1}
\newcommand{\CharTok}[1]{\textcolor[rgb]{0.31,0.60,0.02}{#1}}
\newcommand{\CommentTok}[1]{\textcolor[rgb]{0.56,0.35,0.01}{\textit{#1}}}
\newcommand{\CommentVarTok}[1]{\textcolor[rgb]{0.56,0.35,0.01}{\textbf{\textit{#1}}}}
\newcommand{\ConstantTok}[1]{\textcolor[rgb]{0.00,0.00,0.00}{#1}}
\newcommand{\ControlFlowTok}[1]{\textcolor[rgb]{0.13,0.29,0.53}{\textbf{#1}}}
\newcommand{\DataTypeTok}[1]{\textcolor[rgb]{0.13,0.29,0.53}{#1}}
\newcommand{\DecValTok}[1]{\textcolor[rgb]{0.00,0.00,0.81}{#1}}
\newcommand{\DocumentationTok}[1]{\textcolor[rgb]{0.56,0.35,0.01}{\textbf{\textit{#1}}}}
\newcommand{\ErrorTok}[1]{\textcolor[rgb]{0.64,0.00,0.00}{\textbf{#1}}}
\newcommand{\ExtensionTok}[1]{#1}
\newcommand{\FloatTok}[1]{\textcolor[rgb]{0.00,0.00,0.81}{#1}}
\newcommand{\FunctionTok}[1]{\textcolor[rgb]{0.00,0.00,0.00}{#1}}
\newcommand{\ImportTok}[1]{#1}
\newcommand{\InformationTok}[1]{\textcolor[rgb]{0.56,0.35,0.01}{\textbf{\textit{#1}}}}
\newcommand{\KeywordTok}[1]{\textcolor[rgb]{0.13,0.29,0.53}{\textbf{#1}}}
\newcommand{\NormalTok}[1]{#1}
\newcommand{\OperatorTok}[1]{\textcolor[rgb]{0.81,0.36,0.00}{\textbf{#1}}}
\newcommand{\OtherTok}[1]{\textcolor[rgb]{0.56,0.35,0.01}{#1}}
\newcommand{\PreprocessorTok}[1]{\textcolor[rgb]{0.56,0.35,0.01}{\textit{#1}}}
\newcommand{\RegionMarkerTok}[1]{#1}
\newcommand{\SpecialCharTok}[1]{\textcolor[rgb]{0.00,0.00,0.00}{#1}}
\newcommand{\SpecialStringTok}[1]{\textcolor[rgb]{0.31,0.60,0.02}{#1}}
\newcommand{\StringTok}[1]{\textcolor[rgb]{0.31,0.60,0.02}{#1}}
\newcommand{\VariableTok}[1]{\textcolor[rgb]{0.00,0.00,0.00}{#1}}
\newcommand{\VerbatimStringTok}[1]{\textcolor[rgb]{0.31,0.60,0.02}{#1}}
\newcommand{\WarningTok}[1]{\textcolor[rgb]{0.56,0.35,0.01}{\textbf{\textit{#1}}}}
\usepackage{graphicx}
\makeatletter
\def\maxwidth{\ifdim\Gin@nat@width>\linewidth\linewidth\else\Gin@nat@width\fi}
\def\maxheight{\ifdim\Gin@nat@height>\textheight\textheight\else\Gin@nat@height\fi}
\makeatother
% Scale images if necessary, so that they will not overflow the page
% margins by default, and it is still possible to overwrite the defaults
% using explicit options in \includegraphics[width, height, ...]{}
\setkeys{Gin}{width=\maxwidth,height=\maxheight,keepaspectratio}
% Set default figure placement to htbp
\makeatletter
\def\fps@figure{htbp}
\makeatother
\setlength{\emergencystretch}{3em} % prevent overfull lines
\providecommand{\tightlist}{%
  \setlength{\itemsep}{0pt}\setlength{\parskip}{0pt}}
\setcounter{secnumdepth}{-\maxdimen} % remove section numbering
\ifLuaTeX
  \usepackage{selnolig}  % disable illegal ligatures
\fi

\title{R Basic Building Blocks}
\author{swirl Team}
\date{2022-06-19}

\begin{document}
\maketitle

\begin{Shaded}
\begin{Highlighting}[]
\FunctionTok{library}\NormalTok{(tinytex)}
\end{Highlighting}
\end{Shaded}

\hypertarget{in-this-lesson-we-will-explore-some-basic-building-blocks-of-the-r-programming-language.}{%
\subsubsection{In this lesson, we will explore some basic building
blocks of the R programming
language.}\label{in-this-lesson-we-will-explore-some-basic-building-blocks-of-the-r-programming-language.}}

If at any point you'd like more information on a particular topic
related to R, you can type help.start() at the prompt, which will open a
menu of resources either within RStudio or your default web browser,
depending on your setup). Alternatively, a simple web search often
yields the answer you're looking for.

In its simplest form, R can be used as an interactive calculator. Type 5
+ 7 and press Enter.

\begin{Shaded}
\begin{Highlighting}[]
\DecValTok{5} \SpecialCharTok{+} \DecValTok{7}
\end{Highlighting}
\end{Shaded}

\begin{verbatim}
## [1] 12
\end{verbatim}

R simply prints the result of 12 by default. However, R is a programming
language and often the reason we use a programming language as opposed
to a calculator is to automate some process or avoid unnecessary
repetition.

In this case, we may want to use our result from above in a second
calculation. Instead of retyping 5 + 7 every time we need it, we can
just create a new variable that stores the result.

The way you assign a value to a variable in R is by using the assignment
operator, which is just a `less than' symbol followed by a `minus' sign.
It looks like this: \textless-

Think of the assignment operator as an arrow. You are assigning the
value on the right side of the arrow to the variable name on the left
side of the arrow. To assign the result of 5 + 7 to a new variable
called x, you type x \textless- 5 + 7. This can be read as `x gets 5
plus 7'. Give it a try now.

\begin{Shaded}
\begin{Highlighting}[]
\NormalTok{x }\OtherTok{\textless{}{-}} \DecValTok{5} \SpecialCharTok{+} \DecValTok{7}
\end{Highlighting}
\end{Shaded}

You'll notice that R did not print the result of 12 this time. When you
use the assignment operator, R assumes that you don't want to see the
result immediately, but rather that you intend to use the result for
something else later on.

To view the contents of the variable x, just type x and press Enter. Try
it now.

\begin{Shaded}
\begin{Highlighting}[]
\NormalTok{x}
\end{Highlighting}
\end{Shaded}

\begin{verbatim}
## [1] 12
\end{verbatim}

Now, store the result of x - 3 in a new variable called y.

\begin{Shaded}
\begin{Highlighting}[]
\NormalTok{y }\OtherTok{\textless{}{-}}\NormalTok{ x }\SpecialCharTok{{-}} \DecValTok{3}
\end{Highlighting}
\end{Shaded}

What is the value of y? Type y to find out.

\begin{Shaded}
\begin{Highlighting}[]
\NormalTok{y}
\end{Highlighting}
\end{Shaded}

\begin{verbatim}
## [1] 9
\end{verbatim}

\hypertarget{now-lets-create-a-small-collection-of-numbers-called-a-vector.-any-object-that}{%
\paragraph{Now, let's create a small collection of numbers called a
vector. Any object
that}\label{now-lets-create-a-small-collection-of-numbers-called-a-vector.-any-object-that}}

\hypertarget{contains-data-is-called-a-data-structure-and-numeric-vectors-are-the-simplest}{%
\paragraph{contains data is called a data structure and numeric vectors
are the
simplest}\label{contains-data-is-called-a-data-structure-and-numeric-vectors-are-the-simplest}}

\hypertarget{type-of-data-structure-in-r.-in-fact-even-a-single-number-is-considered-a-vector-of-length-one.}{%
\paragraph{type of data structure in R. In fact, even a single number is
considered a vector of length
one.}\label{type-of-data-structure-in-r.-in-fact-even-a-single-number-is-considered-a-vector-of-length-one.}}

The easiest way to create a vector is with the \textbf{c()} function,
which stands for `concatenate' or `combine'. To create a vector
containing the numbers 1.1, 9, and 3.14, type c(1.1, 9, 3.14). Try it
now and store the result in a variable called z.

\begin{Shaded}
\begin{Highlighting}[]
\NormalTok{z }\OtherTok{\textless{}{-}} \FunctionTok{c}\NormalTok{(}\FloatTok{1.1}\NormalTok{, }\DecValTok{9}\NormalTok{, }\FloatTok{3.14}\NormalTok{)}
\end{Highlighting}
\end{Shaded}

Anytime you have questions about a particular function, you can access
R's built-in help files via the \texttt{?} command. For example, if you
want more information on the \textbf{c()} function, type \textbf{?c}
without the parentheses that normally follow a function name. Give it a
try.

\begin{Shaded}
\begin{Highlighting}[]
\NormalTok{?c}
\end{Highlighting}
\end{Shaded}

\begin{verbatim}
## starting httpd help server ... done
\end{verbatim}

Type z to view its contents. Notice that there are no commas separating
the values in the output.

\begin{Shaded}
\begin{Highlighting}[]
\NormalTok{z}
\end{Highlighting}
\end{Shaded}

\begin{verbatim}
## [1] 1.10 9.00 3.14
\end{verbatim}

You can combine vectors to make a new vector. Create a new vector that
contains z, 555, then z again in that order. Don't assign this vector to
a new variable, cso that we can just see the result immediately.

\begin{Shaded}
\begin{Highlighting}[]
\FunctionTok{c}\NormalTok{(z, }\DecValTok{555}\NormalTok{, z)}
\end{Highlighting}
\end{Shaded}

\begin{verbatim}
## [1]   1.10   9.00   3.14 555.00   1.10   9.00   3.14
\end{verbatim}

Numeric vectors can be used in arithmetic expressions. Type the
following to see what happens: z * 2 + 100.

\begin{Shaded}
\begin{Highlighting}[]
\NormalTok{z }\SpecialCharTok{*} \DecValTok{2} \SpecialCharTok{+} \DecValTok{100}
\end{Highlighting}
\end{Shaded}

\begin{verbatim}
## [1] 102.20 118.00 106.28
\end{verbatim}

First, R multiplied each of the three elements in z by 2. Then it added
100 to each element to get the result you see above. Other common
arithmetic operators are \texttt{+}, \texttt{-}, \texttt{/}, and
\texttt{\^{}} (where x\^{}2 means `x squared'). To take the square root,
use the sqrt() function and to take the absolute value, use the abs()
function.

Take the square root of z - 1 and assign it to a new variable called
my\_sqrt.

\begin{Shaded}
\begin{Highlighting}[]
\NormalTok{my\_sqrt }\OtherTok{\textless{}{-}} \FunctionTok{sqrt}\NormalTok{(z }\SpecialCharTok{{-}}\DecValTok{1}\NormalTok{)}
\end{Highlighting}
\end{Shaded}

Before we view the contents of the my\_sqrt variable, what do you think
it contains?

1: a vector of length 0 (i.e.~an empty vector) 2: a vector of length 3
3: a single number (i.e a vector of length 1)

2

Think about how R handled the other `vectorized' operations:
element-by-element.

Print the contents of my\_sqrt.

\begin{Shaded}
\begin{Highlighting}[]
\NormalTok{my\_sqrt}
\end{Highlighting}
\end{Shaded}

\begin{verbatim}
## [1] 0.3162278 2.8284271 1.4628739
\end{verbatim}

As you may have guessed, R first subtracted 1 from each element of z,
then took the square root of each element. This leaves you with a vector
of the same length as the original vector z.

Now, create a new variable called my\_div that gets the value of z
divided by my\_sqrt.

\begin{Shaded}
\begin{Highlighting}[]
\NormalTok{my\_div }\OtherTok{\textless{}{-}}\NormalTok{ z}\SpecialCharTok{/}\NormalTok{my\_sqrt}
\end{Highlighting}
\end{Shaded}

Which statement do you think is true?

1: my\_div is a single number (i.e a vector of length 1) 2: The first
element of my\_div is equal to the first element of z divided by the
first element of my\_sqrt, and so on\ldots{} 3: my\_div is undefined

Selection: 2

You are quite good my friend!

Go ahead and print the contents of my\_div.

\begin{Shaded}
\begin{Highlighting}[]
\NormalTok{my\_div}
\end{Highlighting}
\end{Shaded}

\begin{verbatim}
## [1] 3.478505 3.181981 2.146460
\end{verbatim}

\hypertarget{arithmetic-operation---etc.-element-by-element}{%
\subsubsection{\texorpdfstring{Arithmetic operation (\texttt{+},
\texttt{-}, \texttt{*}, etc.)
element-by-element}{Arithmetic operation (+, -, *, etc.) element-by-element}}\label{arithmetic-operation---etc.-element-by-element}}

If the vectors are of different lengths, R `recycles' the shorter vector
until it is the same length as the longer vector.

When we did z * 2 + 100 in our earlier example, z was a vector of length
3, but technically 2 and 100 are each vectors of length 1.

Behind the scenes, R is `recycling' the 2 to make a vector of 2s and the
100 to make a vector of 100s. In other words, when you ask R to compute
z * 2 + 100, what it really computes is this: z * c(2, 2, 2) + c(100,
100, 100).

To see another example of how this vector `recycling' works, try adding
c(1, 2, 3, 4) and c(0, 10). Don't worry about saving the result in a new
variable.

\begin{Shaded}
\begin{Highlighting}[]
\FunctionTok{c}\NormalTok{(}\DecValTok{1}\NormalTok{, }\DecValTok{2}\NormalTok{, }\DecValTok{3}\NormalTok{, }\DecValTok{4}\NormalTok{) }\SpecialCharTok{+} \FunctionTok{c}\NormalTok{(}\DecValTok{0}\NormalTok{, }\DecValTok{10}\NormalTok{)}
\end{Highlighting}
\end{Shaded}

\begin{verbatim}
## [1]  1 12  3 14
\end{verbatim}

If the length of the shorter vector does not divide evenly into the
length of the longer vector, R will still apply the `recycling' method,
but will throw a warning to let you know something fishy might be going
on.

Try c(1, 2, 3, 4) + c(0, 10, 100) for an example.

\begin{Shaded}
\begin{Highlighting}[]
\FunctionTok{c}\NormalTok{(}\DecValTok{1}\NormalTok{, }\DecValTok{2}\NormalTok{, }\DecValTok{3}\NormalTok{, }\DecValTok{4}\NormalTok{) }\SpecialCharTok{+} \FunctionTok{c}\NormalTok{(}\DecValTok{0}\NormalTok{, }\DecValTok{10}\NormalTok{, }\DecValTok{100}\NormalTok{)}
\end{Highlighting}
\end{Shaded}

\begin{verbatim}
## Warning in c(1, 2, 3, 4) + c(0, 10, 100): longer object length is not a multiple
## of shorter object length
\end{verbatim}

\begin{verbatim}
## [1]   1  12 103   4
\end{verbatim}

Before concluding this lesson, I'd like to show you a couple of
time-saving tricks.

Earlier in the lesson, you computed z * 2 + 100. Let's pretend that you
made a mistake and that you meant to add 1000 instead of 100. You could
either re-type the expression, or\ldots{}

In many programming environments, the up arrow will cycle through
previous commands. Try hitting the up arrow on your keyboard until you
get to this command (z * 2 + 100), then change 100 to 1000 and hit
Enter. If the up arrow doesn't work for you, just type the corrected
command.

\begin{Shaded}
\begin{Highlighting}[]
\FunctionTok{c}\NormalTok{(}\DecValTok{1}\NormalTok{, }\DecValTok{2}\NormalTok{, }\DecValTok{3}\NormalTok{, }\DecValTok{4}\NormalTok{) }\SpecialCharTok{+} \FunctionTok{c}\NormalTok{(}\DecValTok{0}\NormalTok{, }\DecValTok{10}\NormalTok{, }\DecValTok{1000}\NormalTok{)}
\end{Highlighting}
\end{Shaded}

\begin{verbatim}
## Warning in c(1, 2, 3, 4) + c(0, 10, 1000): longer object length is not a
## multiple of shorter object length
\end{verbatim}

\begin{verbatim}
## [1]    1   12 1003    4
\end{verbatim}

If your environment does not support the up arrow feature, then just
type the corrected command to move on. \#\#\# The following five
functions are probably swirl-specific - not sure! When you are at the R
prompt (\textgreater): -- Typing skip() allows you to skip the current
question. -- Typing play() lets you experiment with R on your own; swirl
will ignore what you do\ldots{} -- UNTIL you type nxt() which will
regain swirl's attention. -- Typing bye() causes swirl to exit. Your
progress will be saved. -- Typing main() returns you to swirl's main
menu. -- Typing info() displays these options again.

In many programming environments, the up arrow will cycle through
previous commands. Try hitting the up arrow on your keyboard until you
get to this command (z * 2 + 100), then change 100 to 1000 and hit
Enter. If the up arrow doesn't work for you, just type the corrected
command.

\end{document}
